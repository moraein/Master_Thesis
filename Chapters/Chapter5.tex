% Chapter Template

\chapter{Summary and Outlook} % Main chapter title

\label{Chapter5} % Change X to a consecutive number; for referencing this chapter elsewhere, use \ref{ChapterX}

In this thesis, a data driven method using tag and probe and fits on the $Z\rightarrow e^+e^-$ events, for the measurement of the misidentification probability of electrons as photons has been presented. The fake rate has been calculated in the full $\sqrt{s}=13$ TeV of $\textit{pp}$ collisions data with $\mathcal{L}=139$ fb$^{-1}$ recorded by the ATLAS between 2015-2018. The fake rate is calculated using two control regions: the $Z\rightarrow ee$ control region, and $Z\rightarrow e\gamma$ control region. The results are presented as functions of kinematic variables of $p_{T}$ and $|\eta|$ with total uncertainties (sum of statistical and systematic). The fake rate distribution was similar for data and MC, ascending from around $2.5\%$ in the central region ($|\eta|< 1.0$) to around $12\%$ for the endcap region ($|\eta|>2.0$). To compare the estimated e-fake contribution by MC simulations with the data, scale factor has been defined as the ratio of fake rate in data over the fake rate in MC. The resulting scale factors revealed values close to unity for most of the $p_{T} - |\eta|$ regions, which means that the $e\rightarrow\gamma$ fake background is well modeled by MC. For the low $p_T$ and $|\eta|$, the scale factors were high up to $1.81\pm0.44$, and the $e\rightarrow\gamma$ fake background in MC needs the correction most. The resulting scale factors for the converted and unconverted photons in the last section of chapter \ref{Chapter4} shown the scale factor for unconverted photons are less dependent on the $\eta$ as compare to the converted photons. The derived Scale Factors from $Z\rightarrow e^+e^-$ decay can be applied to MC based $e\rightarrow\gamma$ fake photon samples in the signal region as a correction factor. As mentioned before, this is based on the assumption that the misidentification of the electrons as photons should be a universal effect and can be corrected by scale factors. The correctness of this assumption can be verified by a comparison between the derived fake rates from $Z\rightarrow e^+e^-$ events and the fake rates from the signal region, in our case the $t\overline{t}\gamma$ signal region.
