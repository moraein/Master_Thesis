% Chapter Template

\chapter{Introduction} % Main chapter title

\label{Introduction} % Change X to a consecutive number; for referencing this chapter elsewhere, use \ref{ChapterX}

The ultimate objective of physics is to describe how nature behaves. The emergence of Quantum Mechanics and General Relativity in the early 20th century completely changed our understanding of the universe. Afterwards, in the mid-1970s, the completion of the Standard Model formulation was a significant success that combines the Classical Field Theory, Quantum Mechanics, and the Special Relativity in one theoretical framework. So far, the Standard Model (SM) of particle physics is one of the most successful theories in physics. It fully describes nature in terms of the elementary particles and their interactions. Similar to any other scientific theory, there are still unanswered questions in the Standard Model, such as the dark matter and neutrino mass. Experiments play many roles in science; one of the essential roles is to test theories and to provide the basis for scientific knowledge \cite{sep-physics-experiment}. The measurements in subatomic scales required an enormous amount of energy. In modern particle physics, this lead to the idea of particle beam colliders that can accelerate the particles near the speed of light and collide them to observe the inner structure of the particles. One of the largest research centers is the European Organization for Nuclear Research (CERN), which is located in Geneva, Switzerland. The Large Hadron Collider (LHC), which is one of the world's most complex experimental facilities is situated at CERN. In 2012 a new particle was observed by the two experiments, ATLAS (A Toroidal LHC ApparatuS) and CMS (Compact Muon Solenoid) at the LHC that turned out to be compatible with the Higgs boson, which was the last prediction of the Standard Model, remained undetected in all previous searches. Since the Higgs boson discovery, one of the main tasks of the LHC is to search for beyond the Standard Model. The top quark is the most massive (strong coupling to the Higgs) standard model particle, and it is unique among all quarks and leptons and might play a role within the mechanism of electroweak symmetry breaking. Measurement of the top quark's coupling to vector bosons has a great importance to test the Standard Model, and any deviation from the Standard Model can give a hint about beyond Standard Model underlying theories. In order to scrutinize the SM, very precise measurements are needed, involving a good understanding of the process under study and the background processes that mimic the signature of the signal. This also requires a good knowledge of the detector performance. One of the contributions to event signatures is the misidentification of physical objects in the detector. In the study of top quark pair production associated with a photon, misidentification of electrons as photons is one of the significant backgrounds. The background is referred to as $e\rightarrow\gamma$ fake. In this thesis, the probability of electrons to be misidentified as photons is investigated using the full 13 TeV data of the ATLAS detector at the LHC at CERN. The description of the theoretical background and the experimental setup is given in Chapters \ref{Chapter2} and \ref{Chapter3}. The analysis is described in Chapter \ref{Chapter4}, and a summary of the results and an outlook is given in Chapter \ref{Chapter5}.
